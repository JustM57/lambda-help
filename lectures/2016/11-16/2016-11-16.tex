
% Default to the notebook output style

    


% Inherit from the specified cell style.




    
\documentclass[11pt]{article}

    
    
    \usepackage[T1]{fontenc}
    % Nicer default font than Computer Modern for most use cases
    \usepackage{palatino}

    % Basic figure setup, for now with no caption control since it's done
    % automatically by Pandoc (which extracts ![](path) syntax from Markdown).
    \usepackage{graphicx}
    % We will generate all images so they have a width \maxwidth. This means
    % that they will get their normal width if they fit onto the page, but
    % are scaled down if they would overflow the margins.
    \makeatletter
    \def\maxwidth{\ifdim\Gin@nat@width>\linewidth\linewidth
    \else\Gin@nat@width\fi}
    \makeatother
    \let\Oldincludegraphics\includegraphics
    % Set max figure width to be 80% of text width, for now hardcoded.
    \renewcommand{\includegraphics}[1]{\Oldincludegraphics[width=.8\maxwidth]{#1}}
    % Ensure that by default, figures have no caption (until we provide a
    % proper Figure object with a Caption API and a way to capture that
    % in the conversion process - todo).
    \usepackage{caption}
    \DeclareCaptionLabelFormat{nolabel}{}
    \captionsetup{labelformat=nolabel}

    \usepackage{adjustbox} % Used to constrain images to a maximum size 
    \usepackage{xcolor} % Allow colors to be defined
    \usepackage{enumerate} % Needed for markdown enumerations to work
    \usepackage{geometry} % Used to adjust the document margins
    \usepackage{amsmath} % Equations
    \usepackage{amssymb} % Equations
    \usepackage{textcomp} % defines textquotesingle
    % Hack from http://tex.stackexchange.com/a/47451/13684:
    \AtBeginDocument{%
        \def\PYZsq{\textquotesingle}% Upright quotes in Pygmentized code
    }
    \usepackage{upquote} % Upright quotes for verbatim code
    \usepackage{eurosym} % defines \euro
    \usepackage[mathletters]{ucs} % Extended unicode (utf-8) support
    \usepackage[utf8x]{inputenc} % Allow utf-8 characters in the tex document
    \usepackage[russian]{babel}
    \usepackage{pscyr}
    \usepackage{fancyvrb} % verbatim replacement that allows latex
    \usepackage{grffile} % extends the file name processing of package graphics 
                         % to support a larger range 
    % The hyperref package gives us a pdf with properly built
    % internal navigation ('pdf bookmarks' for the table of contents,
    % internal cross-reference links, web links for URLs, etc.)
    \usepackage{hyperref}
    \usepackage{longtable} % longtable support required by pandoc >1.10
    \usepackage{booktabs}  % table support for pandoc > 1.12.2
    \usepackage[normalem]{ulem} % ulem is needed to support strikethroughs (\sout)
                                % normalem makes italics be italics, not underlines
    

    
    
    % Colors for the hyperref package
    \definecolor{urlcolor}{rgb}{0,.145,.698}
    \definecolor{linkcolor}{rgb}{.71,0.21,0.01}
    \definecolor{citecolor}{rgb}{.12,.54,.11}

    % ANSI colors
    \definecolor{ansi-black}{HTML}{3E424D}
    \definecolor{ansi-black-intense}{HTML}{282C36}
    \definecolor{ansi-red}{HTML}{E75C58}
    \definecolor{ansi-red-intense}{HTML}{B22B31}
    \definecolor{ansi-green}{HTML}{00A250}
    \definecolor{ansi-green-intense}{HTML}{007427}
    \definecolor{ansi-yellow}{HTML}{DDB62B}
    \definecolor{ansi-yellow-intense}{HTML}{B27D12}
    \definecolor{ansi-blue}{HTML}{208FFB}
    \definecolor{ansi-blue-intense}{HTML}{0065CA}
    \definecolor{ansi-magenta}{HTML}{D160C4}
    \definecolor{ansi-magenta-intense}{HTML}{A03196}
    \definecolor{ansi-cyan}{HTML}{60C6C8}
    \definecolor{ansi-cyan-intense}{HTML}{258F8F}
    \definecolor{ansi-white}{HTML}{C5C1B4}
    \definecolor{ansi-white-intense}{HTML}{A1A6B2}

    % commands and environments needed by pandoc snippets
    % extracted from the output of `pandoc -s`
    \providecommand{\tightlist}{%
      \setlength{\itemsep}{0pt}\setlength{\parskip}{0pt}}
    \DefineVerbatimEnvironment{Highlighting}{Verbatim}{commandchars=\\\{\}}
    % Add ',fontsize=\small' for more characters per line
    \newenvironment{Shaded}{}{}
    \newcommand{\KeywordTok}[1]{\textcolor[rgb]{0.00,0.44,0.13}{\textbf{{#1}}}}
    \newcommand{\DataTypeTok}[1]{\textcolor[rgb]{0.56,0.13,0.00}{{#1}}}
    \newcommand{\DecValTok}[1]{\textcolor[rgb]{0.25,0.63,0.44}{{#1}}}
    \newcommand{\BaseNTok}[1]{\textcolor[rgb]{0.25,0.63,0.44}{{#1}}}
    \newcommand{\FloatTok}[1]{\textcolor[rgb]{0.25,0.63,0.44}{{#1}}}
    \newcommand{\CharTok}[1]{\textcolor[rgb]{0.25,0.44,0.63}{{#1}}}
    \newcommand{\StringTok}[1]{\textcolor[rgb]{0.25,0.44,0.63}{{#1}}}
    \newcommand{\CommentTok}[1]{\textcolor[rgb]{0.38,0.63,0.69}{\textit{{#1}}}}
    \newcommand{\OtherTok}[1]{\textcolor[rgb]{0.00,0.44,0.13}{{#1}}}
    \newcommand{\AlertTok}[1]{\textcolor[rgb]{1.00,0.00,0.00}{\textbf{{#1}}}}
    \newcommand{\FunctionTok}[1]{\textcolor[rgb]{0.02,0.16,0.49}{{#1}}}
    \newcommand{\RegionMarkerTok}[1]{{#1}}
    \newcommand{\ErrorTok}[1]{\textcolor[rgb]{1.00,0.00,0.00}{\textbf{{#1}}}}
    \newcommand{\NormalTok}[1]{{#1}}
    
    % Additional commands for more recent versions of Pandoc
    \newcommand{\ConstantTok}[1]{\textcolor[rgb]{0.53,0.00,0.00}{{#1}}}
    \newcommand{\SpecialCharTok}[1]{\textcolor[rgb]{0.25,0.44,0.63}{{#1}}}
    \newcommand{\VerbatimStringTok}[1]{\textcolor[rgb]{0.25,0.44,0.63}{{#1}}}
    \newcommand{\SpecialStringTok}[1]{\textcolor[rgb]{0.73,0.40,0.53}{{#1}}}
    \newcommand{\ImportTok}[1]{{#1}}
    \newcommand{\DocumentationTok}[1]{\textcolor[rgb]{0.73,0.13,0.13}{\textit{{#1}}}}
    \newcommand{\AnnotationTok}[1]{\textcolor[rgb]{0.38,0.63,0.69}{\textbf{\textit{{#1}}}}}
    \newcommand{\CommentVarTok}[1]{\textcolor[rgb]{0.38,0.63,0.69}{\textbf{\textit{{#1}}}}}
    \newcommand{\VariableTok}[1]{\textcolor[rgb]{0.10,0.09,0.49}{{#1}}}
    \newcommand{\ControlFlowTok}[1]{\textcolor[rgb]{0.00,0.44,0.13}{\textbf{{#1}}}}
    \newcommand{\OperatorTok}[1]{\textcolor[rgb]{0.40,0.40,0.40}{{#1}}}
    \newcommand{\BuiltInTok}[1]{{#1}}
    \newcommand{\ExtensionTok}[1]{{#1}}
    \newcommand{\PreprocessorTok}[1]{\textcolor[rgb]{0.74,0.48,0.00}{{#1}}}
    \newcommand{\AttributeTok}[1]{\textcolor[rgb]{0.49,0.56,0.16}{{#1}}}
    \newcommand{\InformationTok}[1]{\textcolor[rgb]{0.38,0.63,0.69}{\textbf{\textit{{#1}}}}}
    \newcommand{\WarningTok}[1]{\textcolor[rgb]{0.38,0.63,0.69}{\textbf{\textit{{#1}}}}}
    
    
    % Define a nice break command that doesn't care if a line doesn't already
    % exist.
    \def\br{\hspace*{\fill} \\* }
    % Math Jax compatability definitions
    \def\gt{>}
    \def\lt{<}
    % Document parameters
    \title{2016-11-16}
    
    
    

    % Pygments definitions
    
\makeatletter
\def\PY@reset{\let\PY@it=\relax \let\PY@bf=\relax%
    \let\PY@ul=\relax \let\PY@tc=\relax%
    \let\PY@bc=\relax \let\PY@ff=\relax}
\def\PY@tok#1{\csname PY@tok@#1\endcsname}
\def\PY@toks#1+{\ifx\relax#1\empty\else%
    \PY@tok{#1}\expandafter\PY@toks\fi}
\def\PY@do#1{\PY@bc{\PY@tc{\PY@ul{%
    \PY@it{\PY@bf{\PY@ff{#1}}}}}}}
\def\PY#1#2{\PY@reset\PY@toks#1+\relax+\PY@do{#2}}

\expandafter\def\csname PY@tok@mf\endcsname{\def\PY@tc##1{\textcolor[rgb]{0.40,0.40,0.40}{##1}}}
\expandafter\def\csname PY@tok@nd\endcsname{\def\PY@tc##1{\textcolor[rgb]{0.67,0.13,1.00}{##1}}}
\expandafter\def\csname PY@tok@nl\endcsname{\def\PY@tc##1{\textcolor[rgb]{0.63,0.63,0.00}{##1}}}
\expandafter\def\csname PY@tok@o\endcsname{\def\PY@tc##1{\textcolor[rgb]{0.40,0.40,0.40}{##1}}}
\expandafter\def\csname PY@tok@bp\endcsname{\def\PY@tc##1{\textcolor[rgb]{0.00,0.50,0.00}{##1}}}
\expandafter\def\csname PY@tok@mh\endcsname{\def\PY@tc##1{\textcolor[rgb]{0.40,0.40,0.40}{##1}}}
\expandafter\def\csname PY@tok@il\endcsname{\def\PY@tc##1{\textcolor[rgb]{0.40,0.40,0.40}{##1}}}
\expandafter\def\csname PY@tok@nv\endcsname{\def\PY@tc##1{\textcolor[rgb]{0.10,0.09,0.49}{##1}}}
\expandafter\def\csname PY@tok@vg\endcsname{\def\PY@tc##1{\textcolor[rgb]{0.10,0.09,0.49}{##1}}}
\expandafter\def\csname PY@tok@s2\endcsname{\def\PY@tc##1{\textcolor[rgb]{0.73,0.13,0.13}{##1}}}
\expandafter\def\csname PY@tok@no\endcsname{\def\PY@tc##1{\textcolor[rgb]{0.53,0.00,0.00}{##1}}}
\expandafter\def\csname PY@tok@ne\endcsname{\let\PY@bf=\textbf\def\PY@tc##1{\textcolor[rgb]{0.82,0.25,0.23}{##1}}}
\expandafter\def\csname PY@tok@si\endcsname{\let\PY@bf=\textbf\def\PY@tc##1{\textcolor[rgb]{0.73,0.40,0.53}{##1}}}
\expandafter\def\csname PY@tok@sx\endcsname{\def\PY@tc##1{\textcolor[rgb]{0.00,0.50,0.00}{##1}}}
\expandafter\def\csname PY@tok@m\endcsname{\def\PY@tc##1{\textcolor[rgb]{0.40,0.40,0.40}{##1}}}
\expandafter\def\csname PY@tok@ow\endcsname{\let\PY@bf=\textbf\def\PY@tc##1{\textcolor[rgb]{0.67,0.13,1.00}{##1}}}
\expandafter\def\csname PY@tok@c1\endcsname{\let\PY@it=\textit\def\PY@tc##1{\textcolor[rgb]{0.25,0.50,0.50}{##1}}}
\expandafter\def\csname PY@tok@go\endcsname{\def\PY@tc##1{\textcolor[rgb]{0.53,0.53,0.53}{##1}}}
\expandafter\def\csname PY@tok@mb\endcsname{\def\PY@tc##1{\textcolor[rgb]{0.40,0.40,0.40}{##1}}}
\expandafter\def\csname PY@tok@vc\endcsname{\def\PY@tc##1{\textcolor[rgb]{0.10,0.09,0.49}{##1}}}
\expandafter\def\csname PY@tok@gd\endcsname{\def\PY@tc##1{\textcolor[rgb]{0.63,0.00,0.00}{##1}}}
\expandafter\def\csname PY@tok@gu\endcsname{\let\PY@bf=\textbf\def\PY@tc##1{\textcolor[rgb]{0.50,0.00,0.50}{##1}}}
\expandafter\def\csname PY@tok@kn\endcsname{\let\PY@bf=\textbf\def\PY@tc##1{\textcolor[rgb]{0.00,0.50,0.00}{##1}}}
\expandafter\def\csname PY@tok@se\endcsname{\let\PY@bf=\textbf\def\PY@tc##1{\textcolor[rgb]{0.73,0.40,0.13}{##1}}}
\expandafter\def\csname PY@tok@sr\endcsname{\def\PY@tc##1{\textcolor[rgb]{0.73,0.40,0.53}{##1}}}
\expandafter\def\csname PY@tok@s\endcsname{\def\PY@tc##1{\textcolor[rgb]{0.73,0.13,0.13}{##1}}}
\expandafter\def\csname PY@tok@kt\endcsname{\def\PY@tc##1{\textcolor[rgb]{0.69,0.00,0.25}{##1}}}
\expandafter\def\csname PY@tok@nt\endcsname{\let\PY@bf=\textbf\def\PY@tc##1{\textcolor[rgb]{0.00,0.50,0.00}{##1}}}
\expandafter\def\csname PY@tok@cs\endcsname{\let\PY@it=\textit\def\PY@tc##1{\textcolor[rgb]{0.25,0.50,0.50}{##1}}}
\expandafter\def\csname PY@tok@ni\endcsname{\let\PY@bf=\textbf\def\PY@tc##1{\textcolor[rgb]{0.60,0.60,0.60}{##1}}}
\expandafter\def\csname PY@tok@gr\endcsname{\def\PY@tc##1{\textcolor[rgb]{1.00,0.00,0.00}{##1}}}
\expandafter\def\csname PY@tok@gs\endcsname{\let\PY@bf=\textbf}
\expandafter\def\csname PY@tok@vi\endcsname{\def\PY@tc##1{\textcolor[rgb]{0.10,0.09,0.49}{##1}}}
\expandafter\def\csname PY@tok@gh\endcsname{\let\PY@bf=\textbf\def\PY@tc##1{\textcolor[rgb]{0.00,0.00,0.50}{##1}}}
\expandafter\def\csname PY@tok@kc\endcsname{\let\PY@bf=\textbf\def\PY@tc##1{\textcolor[rgb]{0.00,0.50,0.00}{##1}}}
\expandafter\def\csname PY@tok@kr\endcsname{\let\PY@bf=\textbf\def\PY@tc##1{\textcolor[rgb]{0.00,0.50,0.00}{##1}}}
\expandafter\def\csname PY@tok@na\endcsname{\def\PY@tc##1{\textcolor[rgb]{0.49,0.56,0.16}{##1}}}
\expandafter\def\csname PY@tok@cp\endcsname{\def\PY@tc##1{\textcolor[rgb]{0.74,0.48,0.00}{##1}}}
\expandafter\def\csname PY@tok@k\endcsname{\let\PY@bf=\textbf\def\PY@tc##1{\textcolor[rgb]{0.00,0.50,0.00}{##1}}}
\expandafter\def\csname PY@tok@c\endcsname{\let\PY@it=\textit\def\PY@tc##1{\textcolor[rgb]{0.25,0.50,0.50}{##1}}}
\expandafter\def\csname PY@tok@s1\endcsname{\def\PY@tc##1{\textcolor[rgb]{0.73,0.13,0.13}{##1}}}
\expandafter\def\csname PY@tok@sc\endcsname{\def\PY@tc##1{\textcolor[rgb]{0.73,0.13,0.13}{##1}}}
\expandafter\def\csname PY@tok@gi\endcsname{\def\PY@tc##1{\textcolor[rgb]{0.00,0.63,0.00}{##1}}}
\expandafter\def\csname PY@tok@nf\endcsname{\def\PY@tc##1{\textcolor[rgb]{0.00,0.00,1.00}{##1}}}
\expandafter\def\csname PY@tok@gp\endcsname{\let\PY@bf=\textbf\def\PY@tc##1{\textcolor[rgb]{0.00,0.00,0.50}{##1}}}
\expandafter\def\csname PY@tok@kp\endcsname{\def\PY@tc##1{\textcolor[rgb]{0.00,0.50,0.00}{##1}}}
\expandafter\def\csname PY@tok@nb\endcsname{\def\PY@tc##1{\textcolor[rgb]{0.00,0.50,0.00}{##1}}}
\expandafter\def\csname PY@tok@sd\endcsname{\let\PY@it=\textit\def\PY@tc##1{\textcolor[rgb]{0.73,0.13,0.13}{##1}}}
\expandafter\def\csname PY@tok@err\endcsname{\def\PY@bc##1{\setlength{\fboxsep}{0pt}\fcolorbox[rgb]{1.00,0.00,0.00}{1,1,1}{\strut ##1}}}
\expandafter\def\csname PY@tok@mi\endcsname{\def\PY@tc##1{\textcolor[rgb]{0.40,0.40,0.40}{##1}}}
\expandafter\def\csname PY@tok@mo\endcsname{\def\PY@tc##1{\textcolor[rgb]{0.40,0.40,0.40}{##1}}}
\expandafter\def\csname PY@tok@cm\endcsname{\let\PY@it=\textit\def\PY@tc##1{\textcolor[rgb]{0.25,0.50,0.50}{##1}}}
\expandafter\def\csname PY@tok@gt\endcsname{\def\PY@tc##1{\textcolor[rgb]{0.00,0.27,0.87}{##1}}}
\expandafter\def\csname PY@tok@cpf\endcsname{\let\PY@it=\textit\def\PY@tc##1{\textcolor[rgb]{0.25,0.50,0.50}{##1}}}
\expandafter\def\csname PY@tok@ge\endcsname{\let\PY@it=\textit}
\expandafter\def\csname PY@tok@sh\endcsname{\def\PY@tc##1{\textcolor[rgb]{0.73,0.13,0.13}{##1}}}
\expandafter\def\csname PY@tok@kd\endcsname{\let\PY@bf=\textbf\def\PY@tc##1{\textcolor[rgb]{0.00,0.50,0.00}{##1}}}
\expandafter\def\csname PY@tok@nc\endcsname{\let\PY@bf=\textbf\def\PY@tc##1{\textcolor[rgb]{0.00,0.00,1.00}{##1}}}
\expandafter\def\csname PY@tok@ch\endcsname{\let\PY@it=\textit\def\PY@tc##1{\textcolor[rgb]{0.25,0.50,0.50}{##1}}}
\expandafter\def\csname PY@tok@ss\endcsname{\def\PY@tc##1{\textcolor[rgb]{0.10,0.09,0.49}{##1}}}
\expandafter\def\csname PY@tok@sb\endcsname{\def\PY@tc##1{\textcolor[rgb]{0.73,0.13,0.13}{##1}}}
\expandafter\def\csname PY@tok@w\endcsname{\def\PY@tc##1{\textcolor[rgb]{0.73,0.73,0.73}{##1}}}
\expandafter\def\csname PY@tok@nn\endcsname{\let\PY@bf=\textbf\def\PY@tc##1{\textcolor[rgb]{0.00,0.00,1.00}{##1}}}

\def\PYZbs{\char`\\}
\def\PYZus{\char`\_}
\def\PYZob{\char`\{}
\def\PYZcb{\char`\}}
\def\PYZca{\char`\^}
\def\PYZam{\char`\&}
\def\PYZlt{\char`\<}
\def\PYZgt{\char`\>}
\def\PYZsh{\char`\#}
\def\PYZpc{\char`\%}
\def\PYZdl{\char`\$}
\def\PYZhy{\char`\-}
\def\PYZsq{\char`\'}
\def\PYZdq{\char`\"}
\def\PYZti{\char`\~}
% for compatibility with earlier versions
\def\PYZat{@}
\def\PYZlb{[}
\def\PYZrb{]}
\makeatother


    % Exact colors from NB
    \definecolor{incolor}{rgb}{0.0, 0.0, 0.5}
    \definecolor{outcolor}{rgb}{0.545, 0.0, 0.0}



    
    % Prevent overflowing lines due to hard-to-break entities
    \sloppy 
    % Setup hyperref package
    \hypersetup{
      breaklinks=true,  % so long urls are correctly broken across lines
      colorlinks=true,
      urlcolor=urlcolor,
      linkcolor=linkcolor,
      citecolor=citecolor,
      }
    % Slightly bigger margins than the latex defaults
    
    \geometry{verbose,tmargin=1in,bmargin=1in,lmargin=1in,rmargin=1in}
    
    

    \begin{document}
    
    
    \maketitle
    
    

    
    \section{Регулярные выражения в
Python}\label{ux440ux435ux433ux443ux43bux44fux440ux43dux44bux435-ux432ux44bux440ux430ux436ux435ux43dux438ux44f-ux432-python}

    \subsection{Теория:}\label{ux442ux435ux43eux440ux438ux44f}

Регулярные выражения - формальный язык для поиска и манипуляций текстом,
в частности подстроками.

Регулярные выражения основаны на масках (pattern). Это шаблоны или
правила, которые удовлетворяют некоторому множеству строк. Так, из
простых примеров, можно найти все вхождения ``кот'' в строку ``кот
терракот котом котором''.

    Плюсы: + удобны в использовании + универсальны

    Минусы: - регулярные выражения для сложных задач (с множеством условий)
нечитабельны и сложны в разработке - регулярные выражения работают
медленно

    В Python регулярные выражения предоставляются библиотекой \texttt{re}.
Она изначально установлена для всех официальных сборок Python.

Рассмотрим самые часто используемые методы: - \texttt{re.match()} -
\texttt{re.search()} - \texttt{re.findall()} - \texttt{re.split()} -
\texttt{re.sub()} - \texttt{re.compile()}

    \begin{Verbatim}[commandchars=\\\{\}]
{\color{incolor}In [{\color{incolor}1}]:} \PY{k+kn}{import} \PY{n+nn}{re}
        
        \PY{c+c1}{\PYZsh{} Текст, над которым мы будем проводить операции с помощью регулярных выражений}
        \PY{n}{text} \PY{o}{=} \PY{l+s+s2}{\PYZdq{}}\PY{l+s+s2}{The object has the words }\PY{l+s+se}{\PYZbs{}\PYZdq{}}\PY{l+s+s2}{NO STEP}\PY{l+s+se}{\PYZbs{}\PYZdq{}}\PY{l+s+s2}{ on it and could be from the plane}\PY{l+s+s2}{\PYZsq{}}\PY{l+s+s2}{s horizontal stabilizer \PYZhy{} }\PY{l+s+se}{\PYZbs{}}
        \PY{l+s+s2}{        the wing\PYZhy{}like parts attached to the tail, sources say. It was discovered by an American who has been }\PY{l+s+se}{\PYZbs{}}
        \PY{l+s+s2}{        blogging about the search for MH370.}\PY{l+s+s2}{\PYZdq{}}
        
        \PY{n+nb}{print}\PY{p}{(}\PY{l+s+s1}{\PYZsq{}}\PY{l+s+s1}{Text for searching:}\PY{l+s+se}{\PYZbs{}n}\PY{l+s+si}{\PYZob{}0\PYZcb{}}\PY{l+s+s1}{\PYZsq{}}\PY{o}{.}\PY{n}{format}\PY{p}{(}\PY{n}{text}\PY{p}{)}\PY{p}{)}
\end{Verbatim}

    \begin{Verbatim}[commandchars=\\\{\}]
Text for searching:
The object has the words "NO STEP" on it and could be from the plane's horizontal stabilizer -         the wing-like parts attached to the tail, sources say. It was discovered by an American who has been         blogging about the search for MH370.

    \end{Verbatim}

    Рассмотрим методны на простом примере: поиске полного соответствия

\begin{Shaded}
\begin{Highlighting}[]
\NormalTok{re.match(pattern, string)}
\end{Highlighting}
\end{Shaded}

ищет подходящую под маску pattern строку в начале строки text.

    \begin{Verbatim}[commandchars=\\\{\}]
{\color{incolor}In [{\color{incolor}5}]:} \PY{n}{pattern} \PY{o}{=} \PY{l+s+s1}{r\PYZsq{}}\PY{l+s+s1}{The}\PY{l+s+s1}{\PYZsq{}}  
\end{Verbatim}

    \texttt{r} перед строкой указывает, что это ``raw string'' для
регулярного выражения

Почему так см.
https://docs.python.org/3/howto/regex.html\#the-backslash-plague

    \begin{Verbatim}[commandchars=\\\{\}]
{\color{incolor}In [{\color{incolor}10}]:} \PY{n}{result} \PY{o}{=} \PY{n}{re}\PY{o}{.}\PY{n}{match}\PY{p}{(}\PY{n}{pattern}\PY{p}{,} \PY{n}{text}\PY{p}{)}
\end{Verbatim}

    Почему так см.
https://docs.python.org/3/howto/regex.html\#the-backslash-plague

    При успешном поиске будет создан особый объект с результатом, при
неуспешном в result запишется None, то есть ничего. Если попытаться
вывести result - возникнет ошибка

Если найдено, вывести найденный текст, если нет, вывести, что не
найдено.

    \begin{Verbatim}[commandchars=\\\{\}]
{\color{incolor}In [{\color{incolor}11}]:} \PY{n}{result} \PY{o}{=} \PY{n}{result}\PY{o}{.}\PY{n}{group}\PY{p}{(}\PY{l+m+mi}{0}\PY{p}{)} \PY{k}{if} \PY{n}{result} \PY{k}{else} \PY{l+s+s2}{\PYZdq{}}\PY{l+s+s2}{Not found}\PY{l+s+s2}{\PYZdq{}}
         \PY{c+c1}{\PYZsh{} используем метод .group(0) чтобы указать, что хотим получить результат}
         \PY{c+c1}{\PYZsh{} первой группы. О группах позже}
         \PY{n+nb}{print}\PY{p}{(}\PY{l+s+s1}{\PYZsq{}}\PY{l+s+s1}{Searching for }\PY{l+s+se}{\PYZbs{}\PYZdq{}}\PY{l+s+si}{\PYZob{}0\PYZcb{}}\PY{l+s+se}{\PYZbs{}\PYZdq{}}\PY{l+s+s1}{ using match.}\PY{l+s+se}{\PYZbs{}n}\PY{l+s+s1}{Result:}\PY{l+s+se}{\PYZbs{}n}\PY{l+s+si}{\PYZob{}1\PYZcb{}}\PY{l+s+s1}{\PYZsq{}}\PY{o}{.}\PY{n}{format}\PY{p}{(}\PY{n+nb}{str}\PY{p}{(}\PY{n}{pattern}\PY{p}{)}\PY{p}{,} \PY{n+nb}{str}\PY{p}{(}\PY{n}{result}\PY{p}{)}\PY{p}{)}\PY{p}{)}
\end{Verbatim}

    \begin{Verbatim}[commandchars=\\\{\}]
Searching for "The" using match.
Result:
The

    \end{Verbatim}

    Попробуем использовать match для поиска второго слова

Напишем вспомогательную функцию

    \begin{Verbatim}[commandchars=\\\{\}]
{\color{incolor}In [{\color{incolor}13}]:} \PY{k}{def} \PY{n+nf}{result\PYZus{}or\PYZus{}not\PYZus{}found}\PY{p}{(}\PY{n}{result}\PY{p}{)}\PY{p}{:}
             \PY{k}{return} \PY{n}{result}\PY{o}{.}\PY{n}{group}\PY{p}{(}\PY{l+m+mi}{0}\PY{p}{)} \PY{k}{if} \PY{n}{result} \PY{k}{else} \PY{l+s+s2}{\PYZdq{}}\PY{l+s+s2}{Not found}\PY{l+s+s2}{\PYZdq{}}
         
         \PY{n}{pattern} \PY{o}{=} \PY{l+s+s1}{r\PYZsq{}}\PY{l+s+s1}{object}\PY{l+s+s1}{\PYZsq{}}
         
         \PY{n}{result} \PY{o}{=} \PY{n}{result\PYZus{}or\PYZus{}not\PYZus{}found}\PY{p}{(}\PY{n}{re}\PY{o}{.}\PY{n}{match}\PY{p}{(}\PY{n}{pattern}\PY{p}{,} \PY{n}{text}\PY{p}{)}\PY{p}{)}
         
         \PY{n+nb}{print}\PY{p}{(}\PY{l+s+s2}{\PYZdq{}}\PY{l+s+s2}{Searching for }\PY{l+s+se}{\PYZbs{}\PYZdq{}}\PY{l+s+si}{\PYZob{}0\PYZcb{}}\PY{l+s+se}{\PYZbs{}\PYZdq{}}\PY{l+s+s2}{ using match.}\PY{l+s+se}{\PYZbs{}n}\PY{l+s+s2}{Result:}\PY{l+s+se}{\PYZbs{}n}\PY{l+s+si}{\PYZob{}1\PYZcb{}}\PY{l+s+s2}{\PYZdq{}}\PY{o}{.}\PY{n}{format}\PY{p}{(}\PY{n+nb}{str}\PY{p}{(}\PY{n}{pattern}\PY{p}{)}\PY{p}{,} \PY{n+nb}{str}\PY{p}{(}\PY{n}{result}\PY{p}{)}\PY{p}{)}\PY{p}{)}
\end{Verbatim}

    \begin{Verbatim}[commandchars=\\\{\}]
Searching for "object" using match.
Result:
Not found

    \end{Verbatim}

    \texttt{re.search(pattern, string)} похож на \texttt{match()}, но он
ищет не только в начале строки

Повторим опыт с помощью \texttt{search}

    \begin{Verbatim}[commandchars=\\\{\}]
{\color{incolor}In [{\color{incolor}14}]:} \PY{n}{pattern} \PY{o}{=} \PY{l+s+s1}{r\PYZsq{}}\PY{l+s+s1}{The}\PY{l+s+s1}{\PYZsq{}}
         
         \PY{n}{result} \PY{o}{=} \PY{n}{result\PYZus{}or\PYZus{}not\PYZus{}found}\PY{p}{(}\PY{n}{re}\PY{o}{.}\PY{n}{search}\PY{p}{(}\PY{n}{pattern}\PY{p}{,} \PY{n}{text}\PY{p}{)}\PY{p}{)}
         
         \PY{n+nb}{print}\PY{p}{(}\PY{l+s+s2}{\PYZdq{}}\PY{l+s+s2}{Searching for }\PY{l+s+se}{\PYZbs{}\PYZdq{}}\PY{l+s+si}{\PYZob{}0\PYZcb{}}\PY{l+s+se}{\PYZbs{}\PYZdq{}}\PY{l+s+s2}{ using search.}\PY{l+s+se}{\PYZbs{}n}\PY{l+s+s2}{Result:}\PY{l+s+se}{\PYZbs{}n}\PY{l+s+si}{\PYZob{}1\PYZcb{}}\PY{l+s+s2}{\PYZdq{}}\PY{o}{.}\PY{n}{format}\PY{p}{(}\PY{n+nb}{str}\PY{p}{(}\PY{n}{pattern}\PY{p}{)}\PY{p}{,} \PY{n+nb}{str}\PY{p}{(}\PY{n}{result}\PY{p}{)}\PY{p}{)}\PY{p}{)}
\end{Verbatim}

    \begin{Verbatim}[commandchars=\\\{\}]
Searching for "The" using search.
Result:
The

    \end{Verbatim}

    Попробуем использовать search для поиска второго слова

    \begin{Verbatim}[commandchars=\\\{\}]
{\color{incolor}In [{\color{incolor}16}]:} \PY{n}{pattern} \PY{o}{=} \PY{l+s+s1}{r\PYZsq{}}\PY{l+s+s1}{object}\PY{l+s+s1}{\PYZsq{}}
         
         \PY{n}{result} \PY{o}{=} \PY{n}{result\PYZus{}or\PYZus{}not\PYZus{}found}\PY{p}{(}\PY{n}{re}\PY{o}{.}\PY{n}{search}\PY{p}{(}\PY{n}{pattern}\PY{p}{,} \PY{n}{text}\PY{p}{)}\PY{p}{)}
         
         \PY{n+nb}{print}\PY{p}{(}\PY{l+s+s2}{\PYZdq{}}\PY{l+s+s2}{Searching for }\PY{l+s+se}{\PYZbs{}\PYZdq{}}\PY{l+s+si}{\PYZpc{}s}\PY{l+s+se}{\PYZbs{}\PYZdq{}}\PY{l+s+s2}{ using search.}\PY{l+s+se}{\PYZbs{}n}\PY{l+s+s2}{Result:}\PY{l+s+se}{\PYZbs{}n}\PY{l+s+si}{\PYZpc{}s}\PY{l+s+se}{\PYZbs{}n}\PY{l+s+s2}{\PYZdq{}} \PY{o}{\PYZpc{}}
               \PY{p}{(}\PY{n+nb}{str}\PY{p}{(}\PY{n}{pattern}\PY{p}{)}\PY{p}{,} \PY{n+nb}{str}\PY{p}{(}\PY{n}{result}\PY{p}{)}\PY{p}{)}\PY{p}{)}
\end{Verbatim}

    \begin{Verbatim}[commandchars=\\\{\}]
Searching for "object" using search.
Result:
object


    \end{Verbatim}

    В отличие от match мы получили искомую строку.

\begin{center}\rule{3in}{0.4pt}\end{center}

    \texttt{re.findall(pattern, string)} возвращает список всех найденных
совпадений

    \begin{Verbatim}[commandchars=\\\{\}]
{\color{incolor}In [{\color{incolor} }]:} \PY{n}{pattern} \PY{o}{=} \PY{l+s+s1}{r\PYZsq{}}\PY{l+s+s1}{the}\PY{l+s+s1}{\PYZsq{}}
        \PY{n}{result} \PY{o}{=} \PY{n}{re}\PY{o}{.}\PY{n}{findall}\PY{p}{(}\PY{n}{pattern}\PY{p}{,} \PY{n}{text}\PY{p}{)}
        \PY{n+nb}{print}\PY{p}{(}\PY{l+s+s2}{\PYZdq{}}\PY{l+s+s2}{Searching for }\PY{l+s+se}{\PYZbs{}\PYZdq{}}\PY{l+s+si}{\PYZpc{}s}\PY{l+s+se}{\PYZbs{}\PYZdq{}}\PY{l+s+s2}{ using findall.}\PY{l+s+se}{\PYZbs{}n}\PY{l+s+s2}{Result:}\PY{l+s+se}{\PYZbs{}n}\PY{l+s+si}{\PYZpc{}s}\PY{l+s+se}{\PYZbs{}n}\PY{l+s+s2}{\PYZdq{}} \PY{o}{\PYZpc{}}
              \PY{p}{(}\PY{n+nb}{str}\PY{p}{(}\PY{n}{pattern}\PY{p}{)}\PY{p}{,} \PY{n+nb}{str}\PY{p}{(}\PY{n}{result}\PY{p}{)}\PY{p}{)}\PY{p}{)}
\end{Verbatim}

    \texttt{re.split(pattern, string, {[}maxsplit=0{]})} делит строку по
маске \texttt{maxsplit} определяет максимальное количество разделений.
При 0 метод разделит строку столько раз, сколько возможно.

    \begin{Verbatim}[commandchars=\\\{\}]
{\color{incolor}In [{\color{incolor} }]:} \PY{n}{pattern} \PY{o}{=} \PY{l+s+s1}{r\PYZsq{}}\PY{l+s+s1}{the}\PY{l+s+s1}{\PYZsq{}}
        \PY{n}{result} \PY{o}{=} \PY{n}{re}\PY{o}{.}\PY{n}{split}\PY{p}{(}\PY{n}{pattern}\PY{p}{,} \PY{n}{text}\PY{p}{)}
        \PY{n+nb}{print}\PY{p}{(}\PY{l+s+s2}{\PYZdq{}}\PY{l+s+s2}{Splitting text by }\PY{l+s+se}{\PYZbs{}\PYZdq{}}\PY{l+s+si}{\PYZpc{}s}\PY{l+s+se}{\PYZbs{}\PYZdq{}}\PY{l+s+s2}{ using split.}\PY{l+s+se}{\PYZbs{}n}\PY{l+s+s2}{Result:}\PY{l+s+se}{\PYZbs{}n}\PY{l+s+si}{\PYZpc{}s}\PY{l+s+se}{\PYZbs{}n}\PY{l+s+s2}{\PYZdq{}} \PY{o}{\PYZpc{}}
              \PY{p}{(}\PY{n+nb}{str}\PY{p}{(}\PY{n}{pattern}\PY{p}{)}\PY{p}{,} \PY{n+nb}{str}\PY{p}{(}\PY{n}{result}\PY{p}{)}\PY{p}{)}\PY{p}{)}
\end{Verbatim}

    \texttt{re.sub(pattern, repl, string)} ищет маску \texttt{pattern} в
строке \texttt{string} и заменяет её на строку \texttt{repl}

    \begin{Verbatim}[commandchars=\\\{\}]
{\color{incolor}In [{\color{incolor} }]:} \PY{n}{pattern} \PY{o}{=} \PY{l+s+s1}{r\PYZsq{}}\PY{l+s+s1}{NO STEP}\PY{l+s+s1}{\PYZsq{}}
        \PY{n}{repl} \PY{o}{=} \PY{l+s+s1}{\PYZsq{}}\PY{l+s+s1}{LAMBDA}\PY{l+s+s1}{\PYZsq{}}
        \PY{n}{result} \PY{o}{=} \PY{n}{re}\PY{o}{.}\PY{n}{sub}\PY{p}{(}\PY{n}{pattern}\PY{p}{,} \PY{n}{repl}\PY{p}{,} \PY{n}{text}\PY{p}{)}
        \PY{n+nb}{print}\PY{p}{(}\PY{l+s+s2}{\PYZdq{}}\PY{l+s+s2}{Replacing }\PY{l+s+se}{\PYZbs{}\PYZdq{}}\PY{l+s+si}{\PYZpc{}s}\PY{l+s+se}{\PYZbs{}\PYZdq{}}\PY{l+s+s2}{ by }\PY{l+s+se}{\PYZbs{}\PYZdq{}}\PY{l+s+si}{\PYZpc{}s}\PY{l+s+se}{\PYZbs{}\PYZdq{}}\PY{l+s+s2}{ using sub.}\PY{l+s+se}{\PYZbs{}n}\PY{l+s+s2}{Result:}\PY{l+s+se}{\PYZbs{}n}\PY{l+s+si}{\PYZpc{}s}\PY{l+s+se}{\PYZbs{}n}\PY{l+s+s2}{\PYZdq{}} \PY{o}{\PYZpc{}}
              \PY{p}{(}\PY{n+nb}{str}\PY{p}{(}\PY{n}{pattern}\PY{p}{)}\PY{p}{,} \PY{n+nb}{str}\PY{p}{(}\PY{n}{repl}\PY{p}{)}\PY{p}{,} \PY{n+nb}{str}\PY{p}{(}\PY{n}{result}\PY{p}{)}\PY{p}{)}\PY{p}{)}
\end{Verbatim}

    \texttt{re.compile()} создает из строки отдельный объект, который мы
можем использовать для дальнейших операций. Компиляция паттерна
регулярного выражения ускоряет поиск.

    \begin{Verbatim}[commandchars=\\\{\}]
{\color{incolor}In [{\color{incolor} }]:} \PY{n}{pattern} \PY{o}{=} \PY{n}{re}\PY{o}{.}\PY{n}{compile}\PY{p}{(}\PY{l+s+s1}{r\PYZsq{}}\PY{l+s+s1}{the}\PY{l+s+s1}{\PYZsq{}}\PY{p}{)}
        \PY{n}{result} \PY{o}{=} \PY{n}{pattern}\PY{o}{.}\PY{n}{findall}\PY{p}{(}\PY{n}{text}\PY{p}{)}
        \PY{n+nb}{print}\PY{p}{(}\PY{l+s+s2}{\PYZdq{}}\PY{l+s+s2}{Searching for }\PY{l+s+se}{\PYZbs{}\PYZdq{}}\PY{l+s+si}{\PYZpc{}s}\PY{l+s+se}{\PYZbs{}\PYZdq{}}\PY{l+s+s2}{ using findall with compiled str in text1.}\PY{l+s+se}{\PYZbs{}n}\PY{l+s+s2}{Result:}\PY{l+s+se}{\PYZbs{}n}\PY{l+s+si}{\PYZpc{}s}\PY{l+s+se}{\PYZbs{}n}\PY{l+s+s2}{\PYZdq{}} \PY{o}{\PYZpc{}}
              \PY{p}{(}\PY{n+nb}{str}\PY{p}{(}\PY{n}{pattern}\PY{p}{)}\PY{p}{,} \PY{n+nb}{str}\PY{p}{(}\PY{n}{result}\PY{p}{)}\PY{p}{)}\PY{p}{)}
\end{Verbatim}

    \begin{Verbatim}[commandchars=\\\{\}]
{\color{incolor}In [{\color{incolor} }]:} \PY{n}{text2} \PY{o}{=} \PY{l+s+s2}{\PYZdq{}}\PY{l+s+s2}{Early photographic analysis of the object suggests it could have come from the doomed jet,}\PY{l+s+se}{\PYZbs{}}
        \PY{l+s+s2}{         which vanished almost exactly 2 years ago.}\PY{l+s+s2}{\PYZdq{}}
        
        \PY{n}{result} \PY{o}{=} \PY{n}{pattern}\PY{o}{.}\PY{n}{findall}\PY{p}{(}\PY{n}{text2}\PY{p}{)}  \PY{c+c1}{\PYZsh{} Не нужно компилировать паттерн заново}
        
        \PY{n+nb}{print}\PY{p}{(}\PY{l+s+s2}{\PYZdq{}}\PY{l+s+s2}{Searching for }\PY{l+s+se}{\PYZbs{}\PYZdq{}}\PY{l+s+si}{\PYZpc{}s}\PY{l+s+se}{\PYZbs{}\PYZdq{}}\PY{l+s+s2}{ using findall with compiled str in text2.}\PY{l+s+se}{\PYZbs{}n}\PY{l+s+s2}{Result:}\PY{l+s+se}{\PYZbs{}n}\PY{l+s+si}{\PYZpc{}s}\PY{l+s+se}{\PYZbs{}n}\PY{l+s+s2}{\PYZdq{}} \PY{o}{\PYZpc{}}
              \PY{p}{(}\PY{n+nb}{str}\PY{p}{(}\PY{n}{pattern}\PY{p}{)}\PY{p}{,} \PY{n+nb}{str}\PY{p}{(}\PY{n}{result}\PY{p}{)}\PY{p}{)}\PY{p}{)}
\end{Verbatim}

    Пока что в наших паттернах использовались только обычные символы.

``The'' соответствует на языке регулярных выражений только строке
``The''.

Посмотрим на мощный инструмент: метасимволы. Метасимволы это символы,
которые соответстуют особым шаблонам. Вот они.

\begin{itemize}
\itemsep1pt\parskip0pt\parsep0pt
\item
  \texttt{.} Один любой символ, кроме новой строки
  \texttt{\textbackslash{}n}.
\item
  \texttt{?} 0 или 1 вхождение шаблона слева
\item
  \texttt{+} 1 и более вхождений шаблона слева
\item
  \texttt{*} 0 и более вхождений шаблона слева
\item
  \texttt{\textbackslash{}w} Любая цифра или буква
  (\texttt{\textbackslash{}W} --- все, кроме буквы или цифры)
\item
  \texttt{\textbackslash{}d} Любая цифра \texttt{{[}0-9{]}}
  (\texttt{\textbackslash{}D} --- все, кроме цифры)
\item
  \texttt{\textbackslash{}s} Любой пробельный символ
  (\texttt{\textbackslash{}S} --- любой непробельнй символ)
\item
  \texttt{\textbackslash{}b} Граница слова
\item
  \texttt{{[}..{]}} Один из символов в скобках (\texttt{{[}\^{}..{]}}
  --- любой символ, кроме тех, что в скобках)
\item
  \texttt{\textbackslash{}} Экранирование специальных символов
  (\texttt{\textbackslash{}.} означает точку или
  \texttt{\textbackslash{}+} --- знак «плюс»)
\item
  \texttt{\^{}} и \texttt{\$} Начало и конец строки соответственно
\item
  \texttt{\{n,m\}} От \texttt{n} до \texttt{m} вхождений
  (\texttt{\{,m\}} --- от \texttt{0} до \texttt{m})
\item
  \texttt{a\textbar{}b} Соответствует \texttt{a} или \texttt{b}
\item
  \texttt{()} Группирует выражение и возвращает найденный текст
\item
  \texttt{\textbackslash{}t}, \texttt{\textbackslash{}n},
  \texttt{\textbackslash{}r} Символ табуляции, новой строки и возврата
  каретки соответственно
\end{itemize}

\begin{center}\rule{3in}{0.4pt}\end{center}

Примеры использования:

    \begin{Verbatim}[commandchars=\\\{\}]
{\color{incolor}In [{\color{incolor} }]:} \PY{n}{all\PYZus{}symbols} \PY{o}{=} \PY{l+s+s1}{r\PYZsq{}}\PY{l+s+s1}{*}\PY{l+s+s1}{\PYZsq{}}  \PY{c+c1}{\PYZsh{} Соответствует всей строе}
        \PY{c+c1}{\PYZsh{} соответстует одному символу, findall с этим паттерном вернет список}
        \PY{c+c1}{\PYZsh{} символов в строке}
        \PY{n}{symbols} \PY{o}{=} \PY{l+s+s1}{r\PYZsq{}}\PY{l+s+s1}{.}\PY{l+s+s1}{\PYZsq{}}
        \PY{c+c1}{\PYZsh{} соответстует одной букве или цифре, findall с этим паттерном вернет}
        \PY{c+c1}{\PYZsh{} список символов в строке за исключением пробелов}
        \PY{n}{letters\PYZus{}and\PYZus{}numbers} \PY{o}{=} \PY{l+s+s1}{r\PYZsq{}}\PY{l+s+s1}{\PYZbs{}}\PY{l+s+s1}{w}\PY{l+s+s1}{\PYZsq{}}
        \PY{c+c1}{\PYZsh{} findall с этим паттерном вернет список цифр найденных в строке}
        \PY{n}{number} \PY{o}{=} \PY{l+s+s1}{r\PYZsq{}}\PY{l+s+s1}{\PYZbs{}}\PY{l+s+s1}{d}\PY{l+s+s1}{\PYZsq{}}
        \PY{c+c1}{\PYZsh{} findall с этим паттерном вернет список a и an найденных в строке}
        \PY{n}{articules} \PY{o}{=} \PY{l+s+s1}{r\PYZsq{}}\PY{l+s+s1}{a|an}\PY{l+s+s1}{\PYZsq{}}
        \PY{c+c1}{\PYZsh{} findall с этим паттерном вернет список со всеми точками в строке.}
        \PY{c+c1}{\PYZsh{} Заметьте что из\PYZhy{}за экранирования паттерн не соответствует никаким}
        \PY{c+c1}{\PYZsh{} символам, кроме точи}
        \PY{n}{dots} \PY{o}{=} \PY{l+s+s1}{r\PYZsq{}}\PY{l+s+s1}{\PYZbs{}}\PY{l+s+s1}{.}\PY{l+s+s1}{\PYZsq{}}
        \PY{c+c1}{\PYZsh{} findall с этим паттерном вернет список с последним словом в строке}
        \PY{n}{last\PYZus{}word} \PY{o}{=} \PY{l+s+s1}{r\PYZsq{}}\PY{l+s+s1}{\PYZbs{}}\PY{l+s+s1}{w*}\PY{l+s+s1}{\PYZbs{}}\PY{l+s+s1}{.\PYZdl{}}\PY{l+s+s1}{\PYZsq{}}
        \PY{n}{all\PYZus{}words} \PY{o}{=} \PY{l+s+s1}{r\PYZsq{}}\PY{l+s+s1}{\PYZbs{}}\PY{l+s+s1}{w+}\PY{l+s+s1}{\PYZsq{}}  \PY{c+c1}{\PYZsh{} findall с этим паттерном вернет список слов}
        \PY{c+c1}{\PYZsh{} findall с этим паттерном вернет слова, заключенные в кавычки}
        \PY{n}{quoted} \PY{o}{=} \PY{l+s+s1}{r\PYZsq{}}\PY{l+s+s1}{\PYZbs{}}\PY{l+s+s1}{\PYZdq{}}\PY{l+s+s1}{.*}\PY{l+s+s1}{\PYZbs{}}\PY{l+s+s1}{\PYZdq{}}\PY{l+s+s1}{\PYZsq{}}
        
        \PY{c+c1}{\PYZsh{} findall с этим паттерном вернет слова с 5 или более буквами}
        \PY{n}{longwords} \PY{o}{=} \PY{l+s+s1}{r\PYZsq{}}\PY{l+s+s1}{\PYZbs{}}\PY{l+s+s1}{w}\PY{l+s+s1}{\PYZob{}}\PY{l+s+s1}{5,\PYZcb{}}\PY{l+s+s1}{\PYZsq{}}
        
        \PY{c+c1}{\PYZsh{} findall с этим паттерном вернет первые 3 буквы каждого слова}
        \PY{n}{first\PYZus{}three\PYZus{}letters} \PY{o}{=} \PY{l+s+s1}{r\PYZsq{}}\PY{l+s+s1}{\PYZbs{}}\PY{l+s+s1}{b}\PY{l+s+s1}{\PYZbs{}}\PY{l+s+s1}{w}\PY{l+s+si}{\PYZob{}3\PYZcb{}}\PY{l+s+s1}{\PYZsq{}}
        
        \PY{c+c1}{\PYZsh{} findall с этим паттерном вернет слова начинающиеся на a, b или с}
        \PY{n}{starting\PYZus{}with} \PY{o}{=} \PY{l+s+s1}{r\PYZsq{}}\PY{l+s+s1}{\PYZbs{}}\PY{l+s+s1}{b[abc]}\PY{l+s+s1}{\PYZbs{}}\PY{l+s+s1}{w+}\PY{l+s+s1}{\PYZsq{}}
        
        \PY{c+c1}{\PYZsh{} findall с этим паттерном вернет слова не начинающиеся на a, b или с.}
        \PY{c+c1}{\PYZsh{} Обратите внимание на пробел в скобках: он означает, что мы не ищем}
        \PY{c+c1}{\PYZsh{} последовательности символов начинающиеся с пробела.}
        \PY{n}{starting\PYZus{}not\PYZus{}with} \PY{o}{=} \PY{l+s+s1}{r\PYZsq{}}\PY{l+s+s1}{\PYZbs{}}\PY{l+s+s1}{b[\PYZca{}abc ]}\PY{l+s+s1}{\PYZbs{}}\PY{l+s+s1}{w+}\PY{l+s+s1}{\PYZsq{}}
\end{Verbatim}

    \subsubsection{Проверка телефонного
номера}\label{ux43fux440ux43eux432ux435ux440ux43aux430-ux442ux435ux43bux435ux444ux43eux43dux43dux43eux433ux43e-ux43dux43eux43cux435ux440ux430}

    \begin{Verbatim}[commandchars=\\\{\}]
{\color{incolor}In [{\color{incolor} }]:} \PY{n}{li} \PY{o}{=} \PY{p}{[}\PY{l+s+s1}{\PYZsq{}}\PY{l+s+s1}{9999999999}\PY{l+s+s1}{\PYZsq{}}\PY{p}{,} \PY{l+s+s1}{\PYZsq{}}\PY{l+s+s1}{999999\PYZhy{}999}\PY{l+s+s1}{\PYZsq{}}\PY{p}{,} \PY{l+s+s1}{\PYZsq{}}\PY{l+s+s1}{99999x9999}\PY{l+s+s1}{\PYZsq{}}\PY{p}{,} \PY{l+s+s1}{\PYZsq{}}\PY{l+s+s1}{892512303}\PY{l+s+s1}{\PYZsq{}}\PY{p}{,} \PY{l+s+s1}{\PYZsq{}}\PY{l+s+s1}{89293536800}\PY{l+s+s1}{\PYZsq{}}\PY{p}{]}
        \PY{k}{for} \PY{n}{val} \PY{o+ow}{in} \PY{n}{li}\PY{p}{:}
            \PY{k}{if} \PY{n}{re}\PY{o}{.}\PY{n}{match}\PY{p}{(}\PY{l+s+s1}{r\PYZsq{}}\PY{l+s+s1}{[8\PYZhy{}9]}\PY{l+s+si}{\PYZob{}1\PYZcb{}}\PY{l+s+s1}{[0\PYZhy{}9]}\PY{l+s+si}{\PYZob{}9\PYZcb{}}\PY{l+s+s1}{\PYZsq{}}\PY{p}{,} \PY{n}{val}\PY{p}{)} \PY{o+ow}{and} \PY{n+nb}{len}\PY{p}{(}\PY{n}{val}\PY{p}{)} \PY{o}{==} \PY{l+m+mi}{10}\PY{p}{:}
                \PY{n+nb}{print}\PY{p}{(}\PY{k+kc}{True}\PY{p}{)}
            \PY{k}{else}\PY{p}{:}
                \PY{n+nb}{print}\PY{p}{(}\PY{k+kc}{False}\PY{p}{)}
\end{Verbatim}

    \subsection{Источники и дальнейшее
чтение:}\label{ux438ux441ux442ux43eux447ux43dux438ux43aux438-ux438-ux434ux430ux43bux44cux43dux435ux439ux448ux435ux435-ux447ux442ux435ux43dux438ux435}

\begin{itemize}
\itemsep1pt\parskip0pt\parsep0pt
\item
  \href{http://tproger.ru/translations/regular-expression-python/}{Использование
  регулярных выражений в Python для новичков}
\item
  \href{https://habrahabr.ru/post/115825/}{Регулярные выражения, пособие
  для новичков. Часть 1}
\item
  \href{http://pep8.ru/doc/dive-into-python-3/7.html}{Регулярные
  выражения}
\item
  \href{https://docs.python.org/3/howto/regex.html}{Regular Expression
  HOWTO}
\end{itemize}

\subsection{Домашнее
задание}\label{ux434ux43eux43cux430ux448ux43dux435ux435-ux437ux430ux434ux430ux43dux438ux435}

Напишите программу, которая позволяет пользователю ввести с клавиатуры
email и пароль. Проверьте их на следующие правила: - email: - содержит
только латинские буквы, цифры, @ и точку - содержит @ и домен и зону
(.ru, .com и прочее) - домен не короче 3 символов, не длиннее 10
символов, не начинается с цифры - доменная зона не короче двух символов,
не имеет цифр - имя пользователя не длиннее 10 символов, не начинается с
цифры

\begin{itemize}
\itemsep1pt\parskip0pt\parsep0pt
\item
  пароль:

  \begin{itemize}
  \itemsep1pt\parskip0pt\parsep0pt
  \item
    длиннее трех, короче четырех
  \item
    содержит любые символы кроме пробела, таба и переноса строки
  \item
    содержит хотя бы одну латинскую букву, одну цифру, одну латинскую
    букву верхнего регистра
  \item
    не содержит последовательностей букв длиннее трех символов
  \end{itemize}
\end{itemize}

Вам не обязетельно реализовывать все правила в одном регулярном
выражении. Вы можете поступать как удобно, главное чтобы это работало
корректно и вы сами могли понять то, что написали.

    \section{Исключения в
Python}\label{ux438ux441ux43aux43bux44eux447ux435ux43dux438ux44f-ux432-python}

    \begin{Verbatim}[commandchars=\\\{\}]
{\color{incolor}In [{\color{incolor} }]:} \PY{n}{result} \PY{o}{=} \PY{l+m+mi}{1} \PY{o}{/} \PY{l+m+mi}{0}
\end{Verbatim}

    Если запустить этот код мы получим ошибку \texttt{ZeroDivisionError}.
Более корректно называть это исключением.

Существует (как минимум) два различимых вида ошибок: синтаксические
ошибки (\emph{syntax errors}) и исключения (\emph{exceptions}).

Синтаксические ошибки, появляются во время разбора кода интерпретатором.
С точки зрения синтаксиса в коде выше ошибки нет, интерпретатор видит
деление одного integer на другой.

Однако в процессе выполнения возникает исключение. Интерпретатор
разобрал код, но провести операцию не смог. Таким образом ошибки,
обнаруженные при исполнении, называются исключениями
(\emph{exceptions}).

Исключения бывают разных типов и тип исключения выводится в сообщении об
ошибке, например \texttt{ZeroDivisionError}, \texttt{NameError},
\texttt{ValueError}

    \textbf{Давайте обрабатывать!}

Существует возможность написать код, который будет перехватывать
избранные исключения. Посмотрите на представленный пример, в котором
пользователю предлагают вводить число до тех пор, пока оно не окажется
корректным целым. Тем не менее, пользователь может прервать программу
(используя сочетание клавиш Control-C или какое-либо другое,
поддерживаемое операционной системой) Заметьте --- о вызванном
пользователем прерывании сигнализирует исключение KeyboardInterrupt.

    \begin{Verbatim}[commandchars=\\\{\}]
{\color{incolor}In [{\color{incolor} }]:} \PY{k}{while} \PY{k+kc}{True}\PY{p}{:}
            \PY{k}{try}\PY{p}{:}
                \PY{n}{x} \PY{o}{=} \PY{n+nb}{int}\PY{p}{(}\PY{n+nb}{input}\PY{p}{(}\PY{l+s+s2}{\PYZdq{}}\PY{l+s+s2}{Input a number:}\PY{l+s+s2}{\PYZdq{}}\PY{p}{)}\PY{p}{)}
                \PY{k}{break}
            \PY{k}{except} \PY{n+ne}{ValueError}\PY{p}{:}
                \PY{n+nb}{print}\PY{p}{(}\PY{l+s+s2}{\PYZdq{}}\PY{l+s+s2}{Incorrect integer}\PY{l+s+s2}{\PYZdq{}}\PY{p}{)}
\end{Verbatim}

    Оператор \texttt{try} работает следующим образом:

В начале исполняется блок \texttt{try} (операторы между ключевыми
словами \texttt{try} и \texttt{except}). Если при этом не появляется
исключений, блок \texttt{except} не выполняется и оператор \texttt{try}
заканчивает работу. Если во время выполнения блока \texttt{try} было
возбуждено какое-либо исключение, оставшаяся часть блока не выполняется.
Затем, если тип этого исключения совпадает с исключением, указанным
после ключевого слова \texttt{except}, выполняется блок \texttt{except},
а по его завершению выполнение продолжается сразу после оператора
\texttt{try}-\texttt{except}. Если порождается исключение, не
совпадающее по типу с указанным в блоке \texttt{except} --- оно
передаётся внешним операторам \texttt{try}; если ни одного обработчика
не найдено, исключение считается необработанным (\emph{unhandled
exception}), и выполнение полностью останавливается и выводится
сообщение об ошибке.

    Блок \texttt{except} может указывать несколько исключений в виде
заключённого в скобки кортежа.

    \begin{Verbatim}[commandchars=\\\{\}]
{\color{incolor}In [{\color{incolor} }]:} \PY{k}{try}\PY{p}{:}
            \PY{n}{x} \PY{o}{=} \PY{n+nb}{int}\PY{p}{(}\PY{n+nb}{input}\PY{p}{(}\PY{l+s+s2}{\PYZdq{}}\PY{l+s+s2}{Input another number:}\PY{l+s+s2}{\PYZdq{}}\PY{p}{)}\PY{p}{)}
        \PY{k}{except} \PY{p}{(}\PY{n+ne}{RuntimeError}\PY{p}{,} \PY{n+ne}{TypeError}\PY{p}{,} \PY{n+ne}{NameError}\PY{p}{,} \PY{n+ne}{ValueError}\PY{p}{)}\PY{p}{:}
            \PY{n+nb}{print}\PY{p}{(}\PY{l+s+s2}{\PYZdq{}}\PY{l+s+s2}{Caught an exception}\PY{l+s+s2}{\PYZdq{}}\PY{p}{)}
\end{Verbatim}

    В последнем блоке except можно не указывать имени (или имён) исключений.
Тогда он будет действовать как обработчик всех исключений.

    \begin{Verbatim}[commandchars=\\\{\}]
{\color{incolor}In [{\color{incolor} }]:} \PY{k}{while} \PY{k+kc}{True}\PY{p}{:}
            \PY{k}{try}\PY{p}{:}
                \PY{n}{x} \PY{o}{=} \PY{n+nb}{int}\PY{p}{(}\PY{n+nb}{input}\PY{p}{(}\PY{l+s+s2}{\PYZdq{}}\PY{l+s+s2}{Input a number:}\PY{l+s+s2}{\PYZdq{}}\PY{p}{)}\PY{p}{)}
                \PY{k}{break}
            \PY{k}{except}\PY{p}{:}
                \PY{n+nb}{print}\PY{p}{(}\PY{l+s+s2}{\PYZdq{}}\PY{l+s+s2}{Incorrect integer}\PY{l+s+s2}{\PYZdq{}}\PY{p}{)}
\end{Verbatim}

    Получить доступ к информации об исключении можно используя
sys.exc\_info(){[}0{]}

    \begin{Verbatim}[commandchars=\\\{\}]
{\color{incolor}In [{\color{incolor} }]:} \PY{k+kn}{import} \PY{n+nn}{sys}
        \PY{k}{while} \PY{k+kc}{True}\PY{p}{:}
            \PY{k}{try}\PY{p}{:}
                \PY{n}{x} \PY{o}{=} \PY{n+nb}{int}\PY{p}{(}\PY{n+nb}{input}\PY{p}{(}\PY{l+s+s2}{\PYZdq{}}\PY{l+s+s2}{Input a number:}\PY{l+s+s2}{\PYZdq{}}\PY{p}{)}\PY{p}{)}
                \PY{k}{break}
            \PY{k}{except}\PY{p}{:}
                \PY{n+nb}{print}\PY{p}{(}\PY{l+s+s2}{\PYZdq{}}\PY{l+s+s2}{Caught exception:}\PY{l+s+s2}{\PYZdq{}}\PY{p}{,}\PY{n}{sys}\PY{o}{.}\PY{n}{exc\PYZus{}info}\PY{p}{(}\PY{p}{)}\PY{p}{[}\PY{l+m+mi}{0}\PY{p}{]}\PY{p}{)}
\end{Verbatim}

    Более простой способ: записать экземпляр исключения в переменную

    \begin{Verbatim}[commandchars=\\\{\}]
{\color{incolor}In [{\color{incolor} }]:} \PY{k}{while} \PY{k+kc}{True}\PY{p}{:}
            \PY{k}{try}\PY{p}{:}
                \PY{n}{x} \PY{o}{=} \PY{n+nb}{int}\PY{p}{(}\PY{n+nb}{input}\PY{p}{(}\PY{l+s+s2}{\PYZdq{}}\PY{l+s+s2}{Input a number:}\PY{l+s+s2}{\PYZdq{}}\PY{p}{)}\PY{p}{)}
                \PY{k}{break}
            \PY{k}{except} \PY{n+ne}{ValueError} \PY{k}{as} \PY{n}{e}\PY{p}{:} 
                \PY{n+nb}{print}\PY{p}{(}\PY{l+s+s2}{\PYZdq{}}\PY{l+s+s2}{Incorrect integer}\PY{l+s+s2}{\PYZdq{}}\PY{p}{,} \PY{n}{e}\PY{p}{)}
\end{Verbatim}

    Исключения могут охватывать несколько уровней.

При возбуждении исключения оно передается ``вверх'' пока не достигнет
самого высокого уровня или не будет ``поймано'' блоком except.

Это значит, что исключения внутри функции не вызовут ошибку, если
функция будет в блоке try-except:

    \begin{Verbatim}[commandchars=\\\{\}]
{\color{incolor}In [{\color{incolor} }]:} \PY{k}{def} \PY{n+nf}{zero\PYZus{}division}\PY{p}{(}\PY{p}{)}\PY{p}{:}
            \PY{k}{return} \PY{l+m+mi}{1}\PY{o}{/}\PY{l+m+mi}{0}
        
        \PY{k}{try}\PY{p}{:}
            \PY{n}{zero\PYZus{}division}\PY{p}{(}\PY{p}{)}
        \PY{k}{except}\PY{p}{:}
            \PY{n+nb}{print}\PY{p}{(}\PY{l+s+s2}{\PYZdq{}}\PY{l+s+s2}{Caught exception}\PY{l+s+s2}{\PYZdq{}}\PY{p}{)}
\end{Verbatim}

    Можно увеличить вложенность

    \begin{Verbatim}[commandchars=\\\{\}]
{\color{incolor}In [{\color{incolor} }]:} \PY{k}{def} \PY{n+nf}{level\PYZus{}3}\PY{p}{(}\PY{p}{)}\PY{p}{:}
            \PY{k}{return} \PY{l+m+mi}{1}\PY{o}{/}\PY{l+m+mi}{0}
        
        \PY{k}{def} \PY{n+nf}{level\PYZus{}2}\PY{p}{(}\PY{p}{)}\PY{p}{:}
            \PY{k}{return} \PY{n}{level\PYZus{}3}\PY{p}{(}\PY{p}{)}
        
        \PY{k}{def} \PY{n+nf}{level\PYZus{}1}\PY{p}{(}\PY{p}{)}\PY{p}{:}
            \PY{k}{return} \PY{n}{level\PYZus{}2}
        
        \PY{k}{try}\PY{p}{:}
            \PY{n}{level\PYZus{}1}\PY{p}{(}\PY{p}{)}
        \PY{k}{except}\PY{p}{:}
            \PY{n+nb}{print}\PY{p}{(}\PY{l+s+s2}{\PYZdq{}}\PY{l+s+s2}{Caught exception}\PY{l+s+s2}{\PYZdq{}}\PY{p}{)}
\end{Verbatim}

    Можно порождать свои исключения оператором raise

    \begin{Verbatim}[commandchars=\\\{\}]
{\color{incolor}In [{\color{incolor} }]:} \PY{k}{try}\PY{p}{:}
            \PY{k}{raise}\PY{p}{(}\PY{n+ne}{Exception}\PY{p}{(}\PY{l+s+s2}{\PYZdq{}}\PY{l+s+s2}{amazing!}\PY{l+s+s2}{\PYZdq{}}\PY{p}{)}\PY{p}{)}
        \PY{k}{except} \PY{n+ne}{Exception} \PY{k}{as} \PY{n}{e}\PY{p}{:}
            \PY{n+nb}{print}\PY{p}{(}\PY{n}{e}\PY{p}{)}
\end{Verbatim}

    Можно добавить в блок try-except блоки else и finally.

Блок else будет выполнен если try не породил исключений.

Блок finally будет выполнен в любом случае.

    \begin{Verbatim}[commandchars=\\\{\}]
{\color{incolor}In [{\color{incolor} }]:} \PY{k}{def} \PY{n+nf}{divide}\PY{p}{(}\PY{n}{x}\PY{p}{,} \PY{n}{y}\PY{p}{)}\PY{p}{:}
            \PY{k}{try}\PY{p}{:}
                \PY{n}{result} \PY{o}{=} \PY{n}{x} \PY{o}{/} \PY{n}{y}
            \PY{k}{except} \PY{n+ne}{ZeroDivisionError}\PY{p}{:}
                \PY{n+nb}{print}\PY{p}{(}\PY{l+s+s2}{\PYZdq{}}\PY{l+s+s2}{Zero division!}\PY{l+s+s2}{\PYZdq{}}\PY{p}{)}
            \PY{k}{else}\PY{p}{:}
                \PY{n+nb}{print}\PY{p}{(}\PY{l+s+s2}{\PYZdq{}}\PY{l+s+s2}{Result }\PY{l+s+s2}{\PYZdq{}}\PY{p}{,} \PY{n}{result}\PY{p}{)}
            \PY{k}{finally}\PY{p}{:}
                \PY{n+nb}{print}\PY{p}{(}\PY{l+s+s2}{\PYZdq{}}\PY{l+s+s2}{finally}\PY{l+s+s2}{\PYZdq{}}\PY{p}{)}
        
        \PY{n+nb}{print}\PY{p}{(}\PY{n}{divide}\PY{p}{(}\PY{l+m+mi}{1}\PY{p}{,}\PY{l+m+mi}{2}\PY{p}{)}\PY{p}{)}
        \PY{n+nb}{print}\PY{p}{(}\PY{n}{divide}\PY{p}{(}\PY{l+m+mi}{1}\PY{p}{,}\PY{l+m+mi}{0}\PY{p}{)}\PY{p}{)}
\end{Verbatim}

    Можно создавать собственные исключения - для этого нужно объявить новый
класс, наследующийся от Exception.

    \begin{Verbatim}[commandchars=\\\{\}]
{\color{incolor}In [{\color{incolor} }]:} \PY{k}{class} \PY{n+nc}{MyError}\PY{p}{(}\PY{n+ne}{Exception}\PY{p}{)}\PY{p}{:}
            \PY{k}{def} \PY{n+nf}{\PYZus{}\PYZus{}init\PYZus{}\PYZus{}}\PY{p}{(}\PY{n+nb+bp}{self}\PY{p}{,} \PY{n}{value}\PY{p}{)}\PY{p}{:}
                \PY{n+nb+bp}{self}\PY{o}{.}\PY{n}{value} \PY{o}{=} \PY{n}{value}
            \PY{k}{def} \PY{n+nf}{\PYZus{}\PYZus{}str\PYZus{}\PYZus{}}\PY{p}{(}\PY{n+nb+bp}{self}\PY{p}{)}\PY{p}{:}
                \PY{k}{return} \PY{n+nb}{repr}\PY{p}{(}\PY{n+nb+bp}{self}\PY{o}{.}\PY{n}{value}\PY{p}{)}
        
        \PY{k}{try}\PY{p}{:}
            \PY{k}{raise} \PY{n}{MyError}\PY{p}{(}\PY{l+m+mi}{2}\PY{o}{*}\PY{l+m+mi}{2}\PY{p}{)}
        \PY{k}{except} \PY{n}{MyError} \PY{k}{as} \PY{n}{e}\PY{p}{:}
            \PY{n+nb}{print}\PY{p}{(}\PY{l+s+s1}{\PYZsq{}}\PY{l+s+s1}{My exception occurred, value:}\PY{l+s+s1}{\PYZsq{}}\PY{p}{,} \PY{n}{e}\PY{o}{.}\PY{n}{value}\PY{p}{)}
\end{Verbatim}

    Источник: http://pep8.ru/doc/tutorial-3.1/8.html

Для закрепления: \textgreater{} добавить обработку исключений в парсер,
чтобы программа не ``вылетала'' при неудачных попытках читать и писать
несуществующие или заблокированные файлы (исключение IOError например).


    % Add a bibliography block to the postdoc
    
    
    
    \end{document}
